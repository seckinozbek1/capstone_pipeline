\begin{longtable}{llllll}
\caption{Response and Adapted Hypotheses by Question and Response (Sampled)} \label{tab:hypotheses} \\
\toprule
Question & Response & Response Hypothesis & Adapted Hypothesis I & Adapted Hypothesis II & Adapted Hypothesis III \\
\midrule
\endfirsthead
\caption[]{Response and Adapted Hypotheses by Question and Response (Sampled)} \\
\toprule
Question & Response & Response Hypothesis & Adapted Hypothesis I & Adapted Hypothesis II & Adapted Hypothesis III \\
\midrule
\endhead
\midrule
\multicolumn{6}{r}{Continued on next page} \\
\midrule
\endfoot
\bottomrule
\endlastfoot
Q8 & 1 & it is important for children to be independent. & We believe that nurturing the capacity to act autonomously is central to human dignity.   &   In our view, societies flourish when individuals are trusted to make decisions for themselves.   &   Empowering each person to think, choose, and act independently remains a cornerstone of our values.   \\
Q8 & 2 & it is not important for children to be independent. & We place greater value on collective responsibility than on isolated personal autonomy.   &   In our society, interdependence and mutual care take precedence over individual separation.   &   A strong moral compass is shaped not in solitary assertion, but through shared obligations.   \\
Q11 & 1 & it is important for children to be imaginative. & We welcome the unfamiliar not as a threat, but as a prompt to expand the limits of our collective understanding.   &   It is often in the absence of fixed answers that the deepest forms of insight begin to take shape.   &   Our growth as a society depends on those moments when curiosity is permitted to exceed instruction.   \\
Q11 & 2 & it is not important for children to be imaginative. & In our experience, the most reliable outcomes arise from the mastery of what already endures.   &   We find that consistency secures the trust that communities depend upon.   &   Order does not emerge from flights of speculation, but from careful repetition and restraint.   \\
Q17 & 1 & it is important for children to be obedient. & A well-ordered society begins with those who understand when to listen and when to yield.   &   The strength of institutions lies in the ability of individuals to respect boundaries not drawn by themselves.   &   There is merit in restraint, especially when it reflects shared purpose over personal impulse.   \\
Q17 & 2 & it is not important for children to be obedient. & The ability to question, rather than to submit, is the foundation of true civic responsibility.   &   Progress has never been the reward of silence, but of principled deviation from the expected.   &   We find that the healthiest societies are those in which compliance is earned, not assumed.   \\
Q70 & 1 & one has a great deal of confidence in the justice system. & Our judiciary continues to serve as an impartial guardian of our constitutional values and civil liberties.   &   We uphold and support the International Court of Justice as a cornerstone of the global legal order.   &   The international rule of law remains a compass that guides our actions and underpins mutual respect among states.   \\
Q70 & 2 & one has quite a lot of confidence in the justice system. & Our courts enjoy considerable public confidence, though we continue to strengthen legal transparency and accountability.   &   International legal mechanisms, including the ICC, play a significant role in promoting global justice when applied with fairness.   &   Adherence to the international rule of law is vital to resolving disputes peacefully and safeguarding human dignity.   \\
Q70 & 3 & one has some confidence in the justice system. & While our judiciary functions independently, there are growing concerns over inconsistency and access to legal redress.   &   Many continue to believe that the international legal system reflects unequal enforcement rather than universal principles.   &   Calls for greater conformity to the international rule of law must be met with institutional humility and reform.   \\
Q70 & 4 & one has little or no confidence in the justice system. & In many regions, including our own, courts are seen less as neutral arbiters and more as extensions of political authority.   &   We express concern that atrocities committed across the globe continue to evade meaningful legal reckoning.   &   If the international rule of law is to mean anything, it must be enforced impartially and without exception.   \\
\end{longtable}
